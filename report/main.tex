\documentclass[12pt]{article}

% \usepackage{compact}
% \usepackage[a4paper, width=150mm,top=25mm,bottom=25mm]{geometry}
\usepackage[ngerman,english]{babel}
\usepackage[utf8]{inputenc}
\usepackage[T1]{fontenc}
\usepackage[font=small,labelfont=bf,format=hang]{caption}
\usepackage{setspace}
\usepackage{graphicx}
\usepackage[hidelinks]{hyperref}
\usepackage{amsmath}
\usepackage{amssymb}
\usepackage{algorithm}
\usepackage{adjustbox}
\usepackage{algpseudocode}
\usepackage{listings}
\usepackage{booktabs}
\usepackage[rgb]{xcolor}
\usepackage{tikz}
\usetikzlibrary{positioning,fit}
\usepackage{biblatex}
\usepackage[width=0.8\textwidth,labelformat=default]{subcaption}
% \usepackage{fancyhdr}
% \pagestyle{fancy}
\usepackage{url}
% \usepackage[bottom]{footmisc}
\usepackage{bbold}

\definecolor{burgundy}{rgb}{0.5, 0.0, 0.13}
\definecolor{applegreen}{rgb}{0.55, 0.71, 0.0}
\definecolor{lemon}{rgb}{1.0, 0.97, 0.0}
\definecolor{navyblue}{rgb}{0.0, 0.0, 0.5}

\DeclareMathOperator{\BN}{\textit{BN}}
\DeclareMathOperator{\ReLU}{\textit{ReLU}}
\DeclareMathOperator{\KL}{\textit{KL}}
\DeclareMathOperator{\SoftMax}{\textit{SoftMax}}
\DeclareMathOperator{\R}{\mathbb{R}}
\DeclareMathOperator{\Embed}{\text{\textit{Embedding}}}
\DeclareMathOperator*{\argmax}{arg\,max}
\DeclareMathOperator*{\argmin}{arg\,min}
\let\vec\mathbf

\bibliography{references}

\title{Comparison of Approaches to Large Scale Regression}
\author{Till Bungert\\
    \emph{Mat-Nr.: 3302154}\\
\emph{\email{bungert@stud.uni-heidelberg.de}}}

\begin{document}
\maketitle

\begin{abstract}
    Gradient boosting descision trees are the most popular model for tabular
    data regression on kaggle, however they require careful feature
    engineering. We investigate the performance of different methods on such a
    problem with zero feature engineering and find that neural networks perform
    well for this task when paired with embeddings. We also investigate
    differen Gaussian process based approaches, but find that they are harder
    to train than neural networks.
\end{abstract}

\onehalfspacing

\section{Introduction}%
\label{sec:introduction}

Deep neural networks have become the de facto standard tool in many computer
vision tasks, but other methods prevail for different tasks. The most popular
and succesful method for regression on tabular data, at least on Kaggle, is
gradient boosting decision trees~\cite{guo2016entity}.

In this report we look at different and less popular approaches to large scale
regression. We try to estimate a function mapping taxi ride parameters to the
corresponding price for $55M$ records.

We will first investigate neural networks, using entity embeddings for the
categorical inputs as is common practice in neural language processing and has
been applied to a similar problem to ours by De Br{\'e}bisson et
al.~\cite{de2015artificial}.

We will then move on to investigate a more uncommon method and try different
methods of applying Gaussian processes to the problem.

Finally we compare all our methods to gradient boosting decision trees.


\section{Methods}%
\label{sec:methods}

\subsection{Entity Embeddings}%
\label{sub:entity_embeddings}

We map categorical variables with $C$ categories represented by indices $c \in
[0,C)$ to real-numbered vectors $\vec{x}_c \in \R^n$
\begin{equation}
    \Embed: [0,C) \rightarrow \R^n, c \mapsto \Embed(c) = \vec{x}_c.
\end{equation}
These embedding layers are implemented as lookup tables. The vector associated
with each index is a parameter of the model and is learnd jointly with the rest
of the model.

If the input to our model is a mixture of continuous and categorical variables
as is the case here, we learn one embedding layer for each of the categorical
variables and concatenate the vector components of each embedding output
together with the continuous variables to one vector. This concatenated vector
then serves as the input to the rest of the model.

\subsection{Deep Neural Networks}%
\label{sub:deep_neural_networks}

\subsection{Tree-based methods}%
\label{sub:tree_based_methods}

\subsection{Bayesian Optimization}%
\label{sub:bayesian_optimization}


\section{Experiments}%
\label{sec:experiments}

\subsection{Data Set}%
\label{sub:dataset}

\subsection{Model Setups}%
\label{sub:model_setups}

\subsubsection{Simple Neural Network}%
\label{ssub:simple_neural_network}

\subsubsection{Deep Neural Network}%
\label{ssub:deep_neural_network}

\subsubsection{Gaussian Processes}%
\label{ssub:gaussian_processes}

\subsubsection{Gradient Boosting}%
\label{ssub:gradient_boosting}

\subsection{Results}%
\label{sub:results}


\clearpage
\printbibliography

\end{document}
