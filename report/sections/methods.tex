\section{Methods}%
\label{sec:methods}

\subsection{Entity Embeddings}%
\label{sub:entity_embeddings}

We map categorical variables with $C$ categories represented by indices $c \in
\{ 0,1,\dots,C-1\}$ to real-numbered vectors $\vec{x}_c \in \R^n$
\begin{equation}
    \Embed: \{0,1,\dots,C-1\} \rightarrow \R^n, c \mapsto \Embed(c) = \vec{x}_c.
\end{equation}
These embedding layers are implemented as lookup tables. The vector associated
with each index is a parameter of the model and is learnd jointly with the rest
of the model.

If the input to our model is a mixture of continuous and categorical variables
as is the case here, we learn one embedding layer for each of the categorical
variables and concatenate the vector components of each embedding output
together with the continuous variables to one vector. This concatenated vector
then serves as the input to the rest of the model.

\subsection{Neural Networks}%
\label{sub:neural_networks}

\begin{figure}[htb!]
    \def\layersep{2.5cm}
    \centering
    \begin{tikzpicture}[shorten >=1pt,->,draw=black!50, node distance=\layersep]
        \tikzstyle{every pin edge}=[<-,shorten <=1pt]
        \tikzstyle{neuron}=[circle,fill=black!25,minimum size=17pt,inner sep=0pt]
        \tikzstyle{input neuron}=[neuron, fill=green!50];
        \tikzstyle{output neuron}=[neuron, fill=red!50];
        \tikzstyle{hidden neuron}=[neuron, fill=blue!50];
        \tikzstyle{annot} = [text width=4em, text centered]

        % Draw the input layer nodes
        \foreach \name / \y in {1,...,4}
        % This is the same as writing \foreach \name / \y in {1/1,2/2,3/3,4/4}
        \node[input neuron, pin=left:Input \#\y] (I-\name) at (0,-\y) {};

        % Draw the hidden layer nodes
        \foreach \name / \y in {1,...,5}
        \path[yshift=0.5cm]
            node[hidden neuron] (H-\name) at (\layersep,-\y cm) {};

        % Draw the output layer node
        \node[output neuron,pin={[pin edge={->}]right:Output}, right of=H-3] (O) {};

        % Connect every node in the input layer with every node in the
        % hidden layer.
        \foreach \source in {1,...,4}
        \foreach \dest in {1,...,5}
        \path (I-\source) edge (H-\dest);

        % Connect every node in the hidden layer with the output layer
        \foreach \source in {1,...,5}
        \path (H-\source) edge (O);

        % Annotate the layers
        \node[annot,above of=H-1, node distance=1cm] (hl) {Hidden layer};
        \node[annot,left of=hl] {Input layer};
        \node[annot,right of=hl] {Output layer};
    \end{tikzpicture}
    \caption{2-Layer neural network}
    \label{fig:neural_network}
\end{figure}

Feed-forward neural networks, sometimes called multilayer perceptrons, are one
of many machine learning models designed to approximate some function $f^*$.
They define a mapping $y=f(x,\theta)$ where $\theta$ is learned to result in
the best approximation.

The name feed-forward neural network stems from the
fact that they consist of intermediary functions called layers, that are
chained together. The length of the chain of these intermediary functions gives
the depth of the network. As $f^*$, $f$ and all intermediary functions are
vertor valued, the dimensionality of the vector gives the width of the layer.
If a layer is not the input or output layer it is called hidden.
By depicting each vector component as a node, neural networks can be described
by directed acyclic graphs as in \autoref{fig:neural_network}~\cite{Goodfellow-et-al-2016}.

In most cases, each layer consists of a linear function $y=w^T x$, where $w$
are called the weights of that layer and are part of $\theta$. As a chain of
linear functions is still a linear function, we need something to make the
neural network nonlinear to learn general functions. To acomplish this, each
layer is associated with a nonlinear function called the activation function,
that is applied to the result of the linear function. Popular choices for
activation functions are the rectified linear unit (ReLU)
\begin{equation}
    \ReLU(x)=\max(0, x)
\end{equation}
and the softmax~\cite{Goodfellow-et-al-2016}. %TODO

\subsection{Gaussian Processes}%
\label{sub:gaussian_processes}

Gaussian processes are nonparametric~\cite{barberBRML2012}.

\subsection{Tree-based methods}%
\label{sub:tree_based_methods}
